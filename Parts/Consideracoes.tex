% Considerações finais
\chapter{Considerações finais}

As considerações finais formam a parte final (fechamento) do texto, sendo dito de forma resumida (1) o que foi desenvolvido no presente trabalho e quais os resultados do mesmo, (2) o que se pôde concluir após o desenvolvimento bem como as principais contribuições do trabalho, e (3) perspectivas para o desenvolvimento de trabalhos futuros, como listado nos exemplos de seção abaixo. O texto referente às considerações finais do autor deve salientar a extensão e os resultados da contribuição do trabalho e os argumentos utilizados estar baseados em dados comprovados e fundamentados nos resultados e na discussão do texto, contendo deduções lógicas correspondentes aos objetivos do trabalho, propostos inicialmente.


\section{Principais contribuições}

Texto.


\section{Limitações}

Texto.


\section{Trabalhos futuros}

Texto.